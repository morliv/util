Certainly, organizing the equations into more granular groups can help provide a deeper understanding of the relationships between different physical quantities. I'll group the equations based on their underlying principles and interrelations.

### Force and Momentum
1. \( F = ma \) (Newton's Second Law, defining force as the rate of change of momentum)
2. \( p = mv \) (Linear Momentum, linking force and velocity)
3. \( F' = \gamma^3 F \) (Transformation of Force in Special Relativity)

### Work, Energy, and Power
1. \( W = F \cdot d \) (Work, connecting force and displacement)
2. \( KE = \frac{1}{2}mv^2 \) (Kinetic Energy, a specific type of work)
3. \( \Delta U = Q - W \) (First Law of Thermodynamics, relating work and internal energy)
4. \( E = -\frac{Gm_1m_2}{r} \) (Gravitational Potential Energy, a form of energy)
5. \( E = hf \) (Planck's Equation, relating energy and frequency in quantum mechanics)
6. \( E^2 = (mc^2)^2 + (pc)^2 \) (Mass-Energy-Momentum Relation in special relativity)
7. \( PE = mgh \) (Gravitational Potential Energy in a constant gravitational field)
  
### Thermodynamics and Heat
1. \( Q = mc\Delta T \) (Heat Transfer in terms of specific heat)
2. \( W = P\Delta V \) (Work Done in Isothermal Process)
3. \( S = k \ln \Omega \) (Entropy, related to the statistical interpretation of energy)

### Electric and Magnetic Forces
1. \( F = qE \) (Electric Force on a charged particle)
2. \( F = qv \times B \) (Magnetic Force on a moving charged particle)
3. \( V = IR \) (Ohm's Law, connecting voltage, current, and resistance)

### Wave Phenomena
1. \( v = f\lambda \) (Wave Speed in terms of frequency and wavelength)
2. \( n = \frac{c}{v} \) (Index of Refraction)
3. \( n_1 \sin \theta_1 = n_2 \sin \theta_2 \) (Snell's Law, relating angles and indices in refraction)
4. \( \theta_i = \theta_r \) (Law of Reflection)

### Quantum Mechanics
1. \( \Delta x \Delta p \geq \frac{\hbar}{2} \) (Heisenberg Uncertainty Principle)
2. \( i\hbar \frac{\partial \Psi}{\partial t} = H \Psi \) (Time-Dependent Schrödinger Equation)
3. \( H \Psi = E \Psi \) (Time-Independent Schrödinger Equation)

### Cosmology
1. \( H^2 = \frac{8\pi G}{3}\rho - \frac{kc^2}{a^2} \) (Friedmann Equation, governing the expansion of the universe)
2. \( \Omega = \frac{\rho}{\rho_c} \) (Density Parameter, a measure of the matter-energy content of the universe)

### Time and Spacetime
1. \( t' = \frac{t}{\sqrt{1-\frac{v^2}{c^2}}} \) (Time Dilation in special relativity)

This more granular organization should help you see how various equations relate to specific aspects of physical theory. Again, this is by no means an exhaustive list, but it does cover some of the most fundamental and commonly-used equations in physics.